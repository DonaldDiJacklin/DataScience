\documentclass[grey,handout]{beamer}
%\usetheme{Pittsburgh}
\usetheme{Montpellier}
\usepackage{amsmath}
\usepackage{graphicx}
\usepackage{multicol}

\renewcommand{\frametitle}[1]{\begin{center}\textbf{#1}\end{center}}

\def\dd{{\rm d}}
\def\E{\mathbb{E}}
\def\BigO{{\cal O}}

\begin{document}
\title{Is UCB Discriminating Against Females?}

\author{Donald M.~DiJacklin}


\begin{frame}
  \titlepage
\end{frame}

\begin{frame}
\frametitle{ROADMAP OF SEMINAR}
  \begin{enumerate}[<+->]
    \item Goal
    \item Truths
    \item Assumptions
    \item Data
    \item Theory
    \item Results Thus Far
  \end{enumerate}
\end{frame}
\begin{frame}
\frametitle{Goal}
To find the effects changing parameters have on a Ball Python Breeding Program.
\end{frame}
\begin{frame}
\frametitle{Truths}
  \begin{enumerate}[<+->]
    \item Males can inseminate 5 females apiece.
    \item About 60\% of pairings result in a `clutch'.
    \item It costs about \$80 to keep a snake for a year.
    \item A male takes a year to grow to a breedable size.
    \item A female takes two years.
  \end{enumerate}
\end{frame}
\begin{frame}
  \frametitle{Assumptions}
  \begin{enumerate}
    \item Clutch size is distributed discrete triangular [3,14] max 6.
    \item A breeder has a capacity that they are not willing to exceed.
    \item No sickness.
  \end{enumerate}
\end{frame}


\begin{frame}
  % \frametitle{Probability of Clutch}
  % \begin{figure}[H]
  % \centering
  % \includegraphics[width=.5\textwidth]{probECDF.png}
  % \end{figure}
  % 157535.2  186213
\end{frame}

\begin{frame}
  \frametitle{Distribution}
  
\end{frame}


\begin{frame}
  \frametitle{Capacity}
  \begin{multicols}{2}
  % \begin{figure}[H]
  % \centering
  % \includegraphics[width=.5\textwidth]{capECDF.png}
  % \end{figure}
  80799.23 81569.97
  % \begin{figure}[H]
  % \centering
  % \includegraphics[width=.5\textwidth]{othcapECDF.png}
  % \end{figure}
  154631.8 164730.1
  
  \end{multicols}
  
  
  \end{frame}

\end{document}